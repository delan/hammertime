\documentclass[a4paper,12pt,titlepage]{article}
\usepackage[utf8]{inputenc}
\usepackage[T1]{fontenc}
\usepackage[margin=1in]{geometry}
\usepackage{parskip}
\usepackage{graphicx}
\usepackage{hyperref}
\usepackage{listings}
\usepackage[usenames,dvipsnames]{color}
\hypersetup{
	colorlinks,
	pdfauthor=Delan Azabani,
	pdftitle=Computer Communications 200:
	         Stop-and-wait assignment
}
\lstset{basicstyle=\ttfamily, basewidth=0.5em}

\title{Computer Communications 200:\\
       Stop-and-wait assignment}
\date{May 25, 2014}
\author{Azabani, Delan\\17065012\\Wednesday 10 a.m.}

\begin{document}

\maketitle

\section{Organisation of source code}

The source code is divided into the following modules:

\begin{itemize}
	\item \texttt{Makefile} simply contains a \texttt{clean} recipe;
	\item \texttt{TOPOLOGY} defines the compilation options, node topology
	      and link characteristics;
	\item \texttt{assignment.[ch]} contain the handler for
	      \texttt{EV\_REBOOT} as well as global node state;
	\item \texttt{list.[ch]} implement a queue of frames via a
	      double-ended, doubly-linked list;
	\item \texttt{physical.[ch]} deal with data reaching and leaving the
	      physical layer;
	\item \texttt{datalink.[ch]} deal with data reaching the data link
	      layer from above and below;
	\item \texttt{network.[ch]} deal with data reaching the network
	      layer from above and below;
	\item \texttt{application.[ch]} deal with data reaching and leaving
	      the application layer;
	\item \texttt{type.h} contains data structures, constants and other
	      type definitions; and
	\item \texttt{util.h} defines preprocessor macros for assertions,
	      debugging and portable printing.
\end{itemize}

Depending on whether the symbol is a preprocessor macro or C identifier, all
global symbols are prefixed with \texttt{CC200\_} or \texttt{cc200\_}. Doing
this without exception helps avoid polluting the global namespaces as well as
collisions with \texttt{cnet} symbols (such as \texttt{CHECK()} and
\texttt{CC200\_CHECK()}).

\section{Debugging and logging}

By default, even without passing the \texttt{-o} argument to \texttt{cnet},
this assignment logs nodes' messages to three locations concurrently:

\begin{itemize}
	\item the node output window (if using the \texttt{cnet} GUI),
	\item \texttt{stderr} of the invoking pseudo-terminal, and
	\item a unique file named \texttt{hammertime.\$\$.log}.
\end{itemize}

Output via \texttt{CC200\_PRINT()}, which includes \texttt{cnet} errors, are
strictly in the format:

\begin{lstlisting}
(nodeinfo.nodenumber):(__FILE__):(__LINE__):(__FUNC__): message
\end{lstlisting}

Common conventions used within the output include:

\begin{itemize}
	\item \texttt{@0} for node numbers;
	\item \texttt{\%1} for link numbers;
	\item \texttt{\#0} for sequence numbers; and
	\item \texttt{<-} and \texttt{->} for `from' and `to', respectively.
\end{itemize}

The high level of detail and consistency in the assignment's debugging output
significantly aided in finding and correcting defects in the protocol source.

\section{Software reliability}

The protocol source files are compiled with \texttt{-std=c99} to avoid the use
of non-standard language extensions. In the context of static analysis, the
compiler flags \texttt{-Werror -Wall -Wextra -pedantic} are also used to
enforce some best practices.

\texttt{-g} was further enabled to facilitate debugging with \texttt{gdb} and
\texttt{valgrind}. The latter found no invalid memory accesses or leaks other
than any caused by \texttt{cnet}'s own code.

An exception needed to be made with \texttt{-Wno-unused-parameter} --- while
the function arguments to event handlers often remain unused, they must remain
in the declaration to yield a compatible function signature. An alternative
solution includes the use of the \texttt{CC200\_UNUSED()} macro, which is
undesirable.

\section{Simplified program flow}

\texttt{assignment.c} is where the simulation ultimately commences. A static
routing table as well as some state (global to each node) are initialised. The
\texttt{cc200\_link\_timer\_expired} event handler resends the specified link's
data frame waiting at the head of the queue.

The queues and timers, one for each link, are configured. A handful of
diagnostic messages are printed, the two primary events are bound to their
handlers, and the generation of messages by the application layer is commenced.

\begin{enumerate}
	\item \texttt{cc200\_application\_ready} accepts a message of up to 255
	      bytes from the sender's application layer, immediately handing it
	      down to the network layer;
	\item \texttt{cc200\_network\_from\_application} wraps the message in
	      a packet with source and destination node addresses, as well as
	      a field specifying the message's length;
	\item \texttt{cc200\_datalink\_from\_network} wraps the packet in a
	      data frame, tagged with a sequence number and a CRC32 checksum
	      is calculated over the whole frame;
	\item \texttt{cc200\_datalink\_data\_next} sets a timeout for the next
	      data frame to have its corresponding acknowledgement frame
	      received by the sender;
	\item \texttt{cc200\_physical\_from\_datalink} writes the frame to a
	      physical medium;
	\item \texttt{cc200\_physical\_ready} accepts the delivered frame
	      from the physical layer;
	\item \texttt{cc200\_datalink\_from\_physical} verifies the frame's
	      checksum and sequence number, stopping the link's timer and
	      dequeueing a data frame if it is an ACK;
	\item \texttt{cc200\_datalink\_ack} replies to the previous hop with an
	      acknowledgement;
	\item \texttt{cc200\_network\_from\_datalink} either passes the message
	      up to the application layer, or forwards the packet to the next
	      hop in the routing path, as necessary; and
	\item \texttt{cc200\_application\_from\_network} writes the
	      payload to the application layer.
\end{enumerate}

Some functions in this path, such as \texttt{cc200\_datalink\_ack} and
\texttt{cc200\_network\_from\_datalink}, may require further traversal back up
and down this `protocol stack', sometimes or always.

\section{Flow and error control}

For each node, there is a outgoing frame queue for each link. This is required
because multiple application layer messages can be created, or packets arrive
to be forwarded, before an acknowledgement is received on a link. With the
stop-and-wait automatic repeat request method, no additional data frames may
be sent during this time.

This queue is implemented with a linked list allocated on the heap, as opposed
to a circular array buffer, and is thus unbounded in maximum length except by
the memory of the host system. At the same time, however, I did not use
\texttt{CNET\_disable\_application} and \texttt{CNET\_enable\_application} to
create artificial flow control, as this is both unrealistic and virtually
impossible in the real world.

Because the message rate remains constant throughout the simulation process and
packets are never dropped regardless of a frame queue's count, a consequence of
these decisions are that high message rates combined with high propagation
delays and/or low link throughput may lead to congestion and an upward trend in
message delivery time.

\texttt{CNET\_crc32} is used as a fairly reliable and practical scheme for
frame error detection. It is used for both data and acknowledgement frames. The
fact that the function returns \texttt{uint32\_t} instead of writing out a
sequence of bytes was initially disconcerting, as it indicates a lack of
consideration for the issue of endianness, but as \texttt{cnet} always runs on
one system, this is not a significant problem.

To prevent structure alignment padding from yielding unexpected checksum
results and \texttt{sizeof()} values, the \texttt{\_\_attribute\_\_ ((packed))}
type attribute is specified in the definitions of \texttt{cc200\_frame\_t} and
\texttt{cc200\_packet\_t}. While this is not included in the ISO C99 standard,
it is supported by both \texttt{gcc} and \texttt{clang}.

\end{document}
