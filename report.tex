\documentclass[a4paper,12pt,titlepage]{article}
\usepackage[utf8]{inputenc}
\usepackage[T1]{fontenc}
\usepackage{parskip}
\usepackage{graphicx}
\usepackage{hyperref}
\usepackage{listings}
\usepackage[usenames,dvipsnames]{color}
\hypersetup{
	colorlinks,
	pdfauthor=Delan Azabani,
	pdftitle=Computer Communications 200:
	         Stop-and-wait assignment
}
\lstset{basicstyle=\ttfamily, basewidth=0.5em}

\title{Computer Communications 200:\\
       Stop-and-wait assignment}
\date{May 25, 2014}
\author{Delan Azabani\\17065012\\Wednesday 10 a.m.}

\begin{document}

\pagenumbering{gobble}

\maketitle

\pagenumbering{arabic}

\section{Read me}

To interact with this assignment, use the following \texttt{Makefile} rules:

\begin{itemize}
	\item \texttt{make submission} compiles the report and sample logs;
	\item \texttt{make cli} runs the network simulation in a terminal;
	\item \texttt{make gui} runs the network simulation graphically; and
	\item \texttt{make clean} deletes all generated files.
\end{itemize}

\section{Fulfilment of submission requirements}

\begin{enumerate}
	\item The protocol source code can be found in \texttt{*.[ch]};
	\item The test topology file is \texttt{TOPOLOGY};
	\item Sample output for each node is in \texttt{log.node.*.txt}; and
	\item The purposes of the \texttt{README} are served by the report,
	\item Which is the PDF file you are now reading. \texttt{:)}
\end{enumerate}

\section{Organisation of source code}

The source code is divided into the following modules:

\begin{itemize}
	\item \texttt{assignment.[ch]} handles \texttt{EV\_REBOOT} and contains
	      global node state;
	\item \texttt{list.[ch]} implement a queue of frames via a linked list;
	\item \texttt{physical.[ch]} implement the physical layer;
	\item \texttt{datalink.[ch]} implement the data link layer;
	\item \texttt{network.[ch]} implement the network layer;
	\item \texttt{application.[ch]} implement the application layer;
	\item \texttt{type.h} contains data structures, constants and other
	      definitions; and
	\item \texttt{util.h} defines preprocessor macros, some for debugging.
\end{itemize}

All global symbols are prefixed with \texttt{CC200\_} or \texttt{cc200\_} to
avoid namespace pollution, as well as conflicts with \texttt{cnet} symbols such
as \texttt{CHECK()}.

\section{Debugging and logging}

By default, even without passing the \texttt{-o} argument to \texttt{cnet},
this assignment logs nodes' messages to three locations concurrently:

\begin{itemize}
	\item the node output window (if using the \texttt{cnet} GUI),
	\item \texttt{stderr} of the invoking pseudo-terminal, and
	\item a unique file named \texttt{hammertime.\$\$.log}.
\end{itemize}

Output via \texttt{CC200\_PRINT()}, which includes \texttt{cnet} errors, are
strictly in the format:

\begin{lstlisting}
(nodeinfo.nodenumber):(__FILE__):(__LINE__):(__FUNC__): message
\end{lstlisting}

Common conventions used within the output include:

\begin{itemize}
	\item \texttt{@0} for node numbers;
	\item \texttt{\%1} for link numbers;
	\item \texttt{\#0} for sequence numbers; and
	\item \texttt{<-} and \texttt{->} for `from' and `to', respectively.
\end{itemize}

The high level of detail and consistency in the assignment's debugging output
significantly aided in finding and correcting defects in the protocol source.

\section{Software reliability}

The protocol source files are compiled with \texttt{-std=c99} to avoid the use
of non-standard language extensions. In the context of static analysis, the
compiler flags \texttt{-Werror -Wall -Wextra -pedantic} are also used to
enforce some best practices.

\texttt{-g} was further enabled to facilitate debugging with \texttt{gdb} and
\texttt{valgrind}. The latter found no invalid memory accesses or leaks other
than any caused by \texttt{cnet}'s own code.

An exception needed to be made with \texttt{-Wno-unused-parameter} --- while
the function arguments to event handlers often remain unused, they must remain
in the declaration to yield a compatible function signature. An alternative
solution includes the use of the \texttt{CC200\_UNUSED()} macro, which is
undesirable.

\section{Simplified protocol flow}

In \texttt{assignment.c} a static routing table as well as some node-global
state are initialised. The \texttt{cc200\_link\_timer\_expired} event handler
resends the specified link's data frame waiting at the head of the queue.

The queues and timers, one for each link, are configured. The two primary
events are bound to their handlers, and the generation of messages by the
application layer is commenced.

\begin{enumerate}
	\item \texttt{cc200\_application\_ready} accepts a message of up to 255
	      bytes from the sender's application layer, handing it down to
	      the network layer;
	\item \texttt{cc200\_network\_from\_application} wraps the message in
	      a packet with source and destination node addresses, as well as
	      the message length;
	\item \texttt{cc200\_datalink\_from\_network} wraps the packet in a
	      data frame, tagged with a sequence number and a CRC32 checksum;
	\item \texttt{cc200\_datalink\_data\_next} sets a timeout for the next
	      data frame to have its corresponding acknowledgement received by
	      the sender;
	\item \texttt{cc200\_physical\_from\_datalink} physically writes the
	      frame;
	\item \texttt{cc200\_physical\_ready} accepts the physically delivered
	      frame;
	\item \texttt{cc200\_datalink\_from\_physical} verifies the checksum of
	      the frame and its sequence number, stopping the link's timer and
	      dequeueing a pending data frame if it is an acknowledgement;
	\item \texttt{cc200\_datalink\_ack} acknowledges the previous hop;
	\item \texttt{cc200\_network\_from\_datalink} either passes the message
	      up to the application layer, or forwards the packet to the next
	      hop in the routing path, as defined by the source and destination
	      addresses; and
	\item \texttt{cc200\_application\_from\_network} writes to the
	      application layer.
\end{enumerate}

An ideal attempt for single-hop transmission of data may involve:

\begin{itemize}
	\item the sender following steps 1 through 5, then
	\item the recipient following steps 6, 7, 8, 5, 9 and 10, then
	\item the sender following steps 6 and 7.
\end{itemize}

\section{Flow and error control}

For each node, there is a outgoing frame queue for each link. This is required
because multiple application layer messages can be created, or packets arrive
to be forwarded, before an acknowledgement is received on a link. With the
stop-and-wait automatic repeat request method, no more than one unacknowledged
data frame may be sent on each link.

This queue is implemented with a linked list allocated on the heap, as opposed
to a circular array buffer, and is thus unbounded in maximum length except by
the memory of the host system. At the same time, however, I did not use
\texttt{CNET\_disable\_application} and \texttt{CNET\_enable\_application} to
create artificial flow control, as this is unrealistic and impossible in the
real world.

Because the message rate remains constant throughout the simulation process and
packets are never dropped regardless of a frame queue's count, a consequence of
these decisions are that high message rates combined with high propagation
delays and/or low link throughput may lead to congestion and an upward trend in
message delivery time.

\texttt{CNET\_crc32} is used as a fairly reliable and practical scheme for
frame error detection. It is used for both data and acknowledgement frames. The
fact that the function returns \texttt{uint32\_t} instead of writing out a
sequence of bytes was disconcerting, as it indicates a lack of consideration
for endianness, but it should cause no problems as \texttt{cnet} always runs on
one system.

To prevent structure alignment padding from yielding unexpected checksum
results and \texttt{sizeof()} values, the \texttt{\_\_attribute\_\_ ((packed))}
type attribute is specified in the definitions of \texttt{cc200\_frame\_t} and
\texttt{cc200\_packet\_t}. While this type attribute is not included in the ISO
C99 standard, it is supported by both \texttt{gcc} and \texttt{clang}.

\end{document}
